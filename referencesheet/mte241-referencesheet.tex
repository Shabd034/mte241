\documentclass[legalpaper,10pt]{article}

\usepackage{titling}
\usepackage{listings}
\usepackage{url}
\usepackage{setspace}
\usepackage{subfig}
\usepackage{sectsty}
\usepackage{pdfpages}
\usepackage{colortbl}
\usepackage{multirow}
\usepackage{multicol}
\usepackage{relsize}
\usepackage{amsmath}
\usepackage{ifsym}
\usepackage{ulem}
\usepackage{fancyvrb}
\usepackage{amsmath,amssymb,amsthm,graphicx,xspace}
\usepackage[titlenotnumbered,noend,noline]{algorithm2e}
\usepackage[compact]{titlesec}
\usepackage{XCharter}
\usepackage[T1]{fontenc}
\usepackage[scaled]{beramono}
\usepackage{tikz}
\usetikzlibrary{arrows,automata,shapes,trees,matrix,chains,scopes,positioning,calc,patterns}
\tikzstyle{block} = [rectangle, draw, fill=blue!20, 
    text width=2.5em, text centered, rounded corners, minimum height=2em]
\tikzstyle{bw} = [rectangle, draw, fill=blue!20, 
    text width=4em, text centered, rounded corners, minimum height=2em]

\definecolor{namerow}{cmyk}{.40,.40,.40,.40}
\definecolor{namecol}{cmyk}{.40,.40,.40,.40}

\let\LaTeXtitle\title
\renewcommand{\title}[1]{\LaTeXtitle{\textsf{#1}}}

\lstset{basicstyle=\footnotesize\ttfamily,breaklines=true}

\newcommand{\handout}[5]{
  \noindent
  \begin{center}
  \framebox{
    \vbox{
      \hbox to 5.78in { {\bf MTE 241: Introduction to Computer Structures \& Real-Time Systems } \hfill #2 }
      \vspace{4mm}
      \hbox to 5.78in { {\Large \hfill #4  \hfill} }
      \vspace{2mm}
      \hbox to 5.78in { {\em #3 \hfill} }
    }
  }
  \end{center}
  \vspace*{4mm}
}

\newcommand{\lecture}[3]{\handout{#1}{#2}{#3}{Lecture #1}}
\newcommand{\tuple}[1]{\ensuremath{\left\langle #1 \right\rangle}\xspace}

\addtolength{\oddsidemargin}{-1.000in}
\addtolength{\evensidemargin}{-0.500in}
\addtolength{\textwidth}{2.0in}
\addtolength{\topmargin}{-1.000in}
\addtolength{\textheight}{1.75in}
\addtolength{\parskip}{\baselineskip}
\setlength{\parindent}{0in}
\renewcommand{\baselinestretch}{1.5}
\newcommand{\term}{Spring 2015}

\singlespace

\newcommand{\vclass}{MTE 241}
\newcommand{\vtitle}{Function \& Library Reference}
%\newcommand{\vpage}{1}
%\newcommand{\vqcount}{2 }
\newcommand{\rectangle}[3]{\tikz \draw (0,0) rectangle (#1,#2) node[pos=.5]{#3};}
\newcommand{\multiline}[2][c]{%
  \begin{tabular}[#1]{@{}c@{}}#2\end{tabular}}

\usepackage{tikz}
\usepackage{enumitem}
\usepackage{fancyhdr}
 
\pagestyle{fancy}
\usepackage{lastpage}
\lhead{Instructor: J. Zarnett}
\chead{}
\rhead{\bfseries \vtitle: \vclass }
%\lfoot{}
\setlength{\parindent}{0cm}
\setlength{\parskip}{0.3cm}

%\newcommand{\answer}[1]{\fbox{\parbox{.8\linewidth}{\textbf{Answer guide:} \\#1}}}
%\renewcommand{\answer}[1]{}

\begin{document}

\subsubsection*{C Basics}
\vspace{-1em}
\lstset{numbers=none}

\begin{lstlisting}[language=C]
void* malloc( int num_bytes );
void* calloc( size_t num, size_t size );
void *realloc( void *ptr, size_t new_size );
void free( void* p );
int atoi( const char *s ); /* Convert string to int */
int printf( const char * fmt, ... ); /* %d int, %lu unsigned long, %f double,  %s string, \n newline, ...=args */
void* memcpy( void *destination, const void *source, size_t num_bytes ); /* Returns destination */
void *memset( void *mem, int value, size_t num_bytes ); /* Set memory to 0 */
size_t strlen( const char * string ); /* returns length of null-terminated string, not counting the terminator */
int strncmp ( const char * str1, const char * str2, size_t max_size ); /* return 0 if strings equal */
\end{lstlisting}
\vspace{-1em}

\subsubsection*{Files}
\vspace{-1em}
\begin{lstlisting}[language=C]
int open( const char *filename, int flags );  /* Returns a file descriptor if successful, -1 on error */
ssize_t read( int file_descriptor, void *buffer, size_t count ); /* Returns number of bytes read */
ssize_t write( int file_descriptor, const void *buffer, size_t count ); /* Returns number of bytes written */
int rename( const char *old_filename, const char *new_filename ); /* Returns 0 on success */
int close( int file_descriptor ); 
FILE* fopen( const char *filename, const char *mode );
int fclose( FILE* f );
size_t fread(void *restrict ptr, size_t size, size_t nmemb, FILE *restrict stream);
size_t fwrite(const void *restrict ptr, size_t size, size_t nmemb, FILE *restrict stream);
int fprintf( FILE* f, const char* format, ... ); /* Arguments go in the ...  */
int fscanf( FILE* f, const char* format, ... ); /* Arguments go in the ...  */
int fileno( FILE* f ); /* Convert FILE* to file descriptor */
int fseek ( FILE * f, long int offset, int origin ); /*SEEK_SET from file start. SEEK_CUR from current loc.*/
int remove( const char* filename );

void* mmap( void* address, size_t length, int protection, int flag, int fd, off_t offset ); /* address = NULL */
int mprotect( void* address, size_t length, int prot ); /* PROT_NONE, PROT_READ, PROT_WRITE, PROT_EXECUTE */
int msync( void* address, size_t length, int flags ); /* Use MS_SYNC for flags */
int munmap( void* address, size_t length );
\end{lstlisting}

When opening a file with \texttt{fopen()}, the options are:\\
\begin{tabular}{l|l}
	\textbf{Mode} & \textbf{Meaning} \\ \hline
	\texttt{r} & Open the file read-only. \\ \hline
	\texttt{w} & Open the file for writing (create if needed) \\ \hline
	\texttt{a} & Open the file for appending (create if needed) \\ \hline
	\texttt{r+} & Open the file for reading, from the start \\ \hline
	\texttt{w+} & Open the file for writing (overwrite) \\ \hline
	\texttt{a+} & Open the file for reading and writing, append if exists \\
\end{tabular}


For \texttt{open()} the following flags may be used for the \texttt{flags} parameter, and can be combined with \texttt{|} (bitwise OR):\\
\begin{tabular}{l|l}
	\textbf{Flag} & \textbf{Meaning} \\ \hline
	\texttt{O\_RDONLY} & Open the file read-only \\ \hline
	\texttt{O\_WRONLY} & Open the file write-only \\ \hline
	\texttt{O\_RDWR} & Open the file for both reading and writing \\ \hline
	\texttt{O\_APPEND} & Append information to the end of the file \\ \hline
	\texttt{O\_TRUNC} & Initially clear all data from the file \\ \hline
	\texttt{O\_CREAT} & Create the file \\ \hline	
	\texttt{O\_EXCL} & If used with \texttt{O\_CREAT}, the caller MUST create the file; if the file exists it will fail \\ 	
\end{tabular}


\subsubsection*{Processes, Threads, Concurrency}
\vspace{-1em}
\begin{lstlisting}[language=C]
pid_t fork( );
pid_t wait( int* status );
pid_t waitpid( pid_t pid, int *status, int options ); /* 0 for options fine */
void exit( int status );

pthread_create( pthread_t *thread, const pthread_attr_t *attr, void *(*start_routine)( void * ), void *arg );
pthread_join( pthread_t thread, void **returnValue );
pthread_detach( pthread_t thread );
pthread_cancel( pthread_t thread );
pthread_testcancel( ); /* If the thread is cancelled, this function does not return (thread terminated) */
pthread_setcanceltype( int type, int *oldtype ); /* PTHREAD_CANCEL_DEFERRED or PTHREAD_CANCEL_ASYNCHRONOUS */
pthread_cleanup_push( void (*routine)(void*), void *argument ); /* Register cleanup handler, with argument */ 
pthread_cleanup_pop( int execute ); /* Run if execute is non-zero */
pthread_exit( void *value );

pthread_mutex_init( pthread_mutex_t *mutex, pthread_mutexattr_t *attributes );
pthread_mutex_lock( pthread_mutex_t *mutex );
pthread_mutex_trylock( pthread_mutex_t *mutex ); /* Returns 0 on success */
pthread_mutex_unlock( pthread_mutex_t *mutex );
pthread_mutex_destroy( pthread_mutex_t *mutex );
pthread_rwlock_init( pthread_rwlock_t * rwlock, pthread_rwlockattr_t * attr );
pthread_rwlock_rdlock( pthread_rwlock_t * rwlock );
pthread_rwlock_tryrdlock( pthread_rwlock_t * rwlock );
pthread_rwlock_wrlock( pthread_rwlock_t * rwlock );
pthread_rwlock_trywrlock( pthread_rwlock_t * rwlock );
pthread_rwlock_unlock( pthread_rwlock_t * rwlock );
pthread_rwlock_destroy( pthread_rwlock_t * rwlock );

sem_init( sem_t* semaphore, int shared, int initial_value); /* 0 for shared OK */
sem_destroy( sem_t* semaphore );
sem_wait( sem_t* semaphore );
sem_trywait( semt_t* semaphore );
sem_post( sem_t* semaphore );
\end{lstlisting}

\end{document}